% In this file you should put the actual content of the blueprint.
% It will be used both by the web and the print version.
% It should *not* include the \begin{document}
%
% If you want to split the blueprint content into several files then
% the current file can be a simple sequence of \input. Otherwise It
% can start with a \section or \chapter for instance.
\newcommand*{\unitball}[1]{{#1}^\circ}
\newcommand*{\ZZ}{\mathbb{Z}}
\newcommand*{\Qp}[1][p]{\mathbb{Q}_{#1}}
\newcommand*{\Fp}[1][p]{\mathbb{F}_{#1}}
\newcommand*{\laurentseries}[1][\Fp]{{#1}(\!(X)\!)}
\newcommand*{\primeid}[1][p]{\mathfrak{#1}}

\chapter{Test}

\section{Generalities}
Fix a field $K$.
\begin{c}
	\label{def:discrete-val}
	\lean{is_discrete}
	\leanok
	A valuation $v\colon K \to \ZZ$ on $K$ is discrete if it is non-archimedean and surjective.
\end{def}
We have the

\begin{proposition}[{\cite[Chap I,\S1, Proposition 1]{Se2}}]
	\label{prop:unit_ball_dvr}
	\lean{discrete_valuation.integer.discrete_valuation_ring}
	If $K$ is complete with respect to a discrete valuation $v$, then its unit ball $\unitball{K}$ is a discrete valuation ring.
\end{proposition}

\begin{lemma}[{\cite[Chap II,\S2, last part of Proposition 3]{Se2}}] 
	\label{lemma:val_L}
	\lean{discrete_valuation.hw}
	\lean{discrete_valuation.complete_of_finite}
	\lean{discrete_valuation.is_discrete_of_finite}
	If $K$ is complete with respect to a discrete valuation $v$ and if $L/K$ is a finite extension, then $L$ has a discrete valuation $w\colon L \twoheadrightarrow \ZZ$ inducing $v$ and $L$ is complete with respect to $w$.
\end{lemma}

Moreover,
\begin{proposition}[{\cite[Chap II,\S2, Proposition 3]{Se2}}]
	\label{prop:unitball_L}
	\lean{discrete_valuation.integral_closure_eq_integer}
	\lean{discrete_valuation.integral_closure.discrete_valuation_ring}
	\lean{discrete_valuation.integral_closure_finrank}
	If $K$ is complete with respect to a discrete valuation $v$ and if $L/K$ is a finite extension, then the integral closure of $\unitball{K}$ inside $L$ coincides $\unitball{L}$ and so, in particular, it is a discrete valuation ring by~\ref{prop:unit_ball_dvr}. Moreover, $\unitball{L}$ is a finite, free $\unitball{K}$-module of rank $n=[L:K]$.
\end{proposition}
\section{Local Fields}
\begin{def}
	\label{local_field}
	\lean{local_field}
	\leanok
	A \textit{(nonarchimedean) local field} is a field complete with respect to a discrete
	valuation and with finite residue field.
\end{def}

\begin{def}
	\label{mixed_char_local_field}
	\lean{mixed_char_local_field}
	\leanok
	A \textit{mixed characteristic local field} is a finite field extension of the field 
	$\Qp$ of $p$-adic numbers, for some prime $p$.
\end{def}

\begin{def}
	\label{eq_char_local_field}
	\lean{eq_char_local_field}
	\leanok
	An \textit{equal characteristic local field} is a finite field extension of the field 
	$\laurentseries$, for some prime $p$.
\end{def}

\begin{lemma}
	\label{mixed_char_local_field.local_field}
	\lean{mixed_char_local_field.local_field}
	\uses{local_field, mixed_char_local_field}
	A mixed characteristic local field is a local field.
\end{lemma}

\begin{lemma}
	\label{eq_char_local_field.local_field}
	\lean{eq_char_local_field.local_field}
	\uses{local_field, eq_char_local_field}
	An equal characteristic local field is a local field.
\end{lemma}
\subsection{Ramification Index}
\begin{enumerate}
	\item Define a local ramification index to get rid of the variables $\primeid$ and $\primeid[P]$ in the mathlib definition
	\item Obtain from there the formula $n=ef$
	\item Prove that $\Qp$ and $\laurentseries$ are unramified
	\item If possible, go through \cite[Chap. I, \S6]{Se2} (before completion, the results there hold for DVR's).
\end{enumerate}
\section{Global to Local}
Starting with a Dedekind domain $R$ and a non-zero maximal ideal $\primeid$, let $K=\operatorname{Frac}(R)$. Then we prove the
\begin{proposition}[{\cite[Chap. II, \S2, Théorème 1]{Se2}}]
	\label{prop:extension_equal}
	\lean{is_dedekind_domain.height_one_spectrum.is_discrete}
	The completion $K_{\primeid}$ has a discrete valuation extending the valuation $v_{\primeid}$ and such that $\unitball{K_{\primeid}}=R_{\primeid}$. In particular, this localization is a DVR by Proposition~\ref{prop:unit_ball_dvr}.
\end{proposition}

\begin{proposition}
	\label{prop:number_field_localization}
	\lean{number_field.adic_completion.mixed_char_local_field}
	Let $F$ be a number field and let $v$ be a finite place of residue characteristic $p$. Then $F_v$ is a mixed characteristic local field with residue characteristic $p$.
\end{proposition}

\begin{proposition}
	\label{prop:function_field_localization}
	\lean{function_field.adic_completion.eq_char_local_field}
	Let $F$ be a function field and let $v$ be any place. Then $F_v$ is an equal characteristic local field with residue characteristic $p$.
\end{proposition}
